% -*- Mode:TeX -*-

%% IMPORTANT: The official thesis specifications are available at:
%%            http://libraries.mit.edu/archives/thesis-specs/
%%
%%            Please verify your thesis' formatting and copyright
%%            assignment before submission.  If you notice any
%%            discrepancies between these templates and the
%%            MIT Libraries' specs, please let us know
%%            by e-mailing thesis@mit.edu

%% The documentclass options along with the pagestyle can be used to generate
%% a technical report, a draft copy, or a regular thesis.  You may need to
%% re-specify the pagestyle after you \include  cover.tex.  For more
%% information, see the first few lines of mitthesis.cls.

%\documentclass[12pt,vi,twoside]{mitthesis}
%%
%%  If you want your thesis copyright to you instead of MIT, use the
%%  ``vi'' option, as above.
%%
%\documentclass[12pt,twoside,leftblank]{mitthesis}
%%
%% If you want blank pages before new chapters to be labelled ``This
%% Page Intentionally Left Blank'', use the ``leftblank'' option, as
%% above.

\documentclass[12pt,twoside]{mitthesis}
\usepackage{lgrind}
%% These have been added at the request of the MIT Libraries, because
%% some PDF conversions mess up the ligatures.  -LB, 1/22/2014
\usepackage{cmap}
\usepackage[T1]{fontenc}
\usepackage[utf8]{inputenc}
\usepackage{textcomp}
% \usepackage[utf8x]{inputenc}
\usepackage{amssymb,amsmath}
\usepackage{inconsolata}

$if(natbib)$
    \usepackage{natbib}
    \bibliographystyle{$if(biblio-style)$$biblio-style$$else$plainnat$endif$}
$endif$
$if(biblatex)$
    \usepackage{biblatex}
$for(bibliography)$
    \addbibresource{$bibliography$}
$endfor$
$endif$
$if(highlighting-macros)$
    $highlighting-macros$
$endif$
$if(verbatim-in-note)$
    \usepackage{fancyvrb}
    \VerbatimFootnotes
$endif$
$if(tables)$
    \usepackage{longtable,booktabs}
$endif$
$if(graphics)$
    \usepackage{graphicx,grffile}
    \makeatletter
    \def\maxwidth{\ifdim\Gin@nat@width>\linewidth\linewidth\else\Gin@nat@width\fi}
    \def\maxheight{\ifdim\Gin@nat@height>\textheight\textheight\else\Gin@nat@height\fi}
    \makeatother
    % Scale images if necessary, so that they will not overflow the page
    % margins by default, and it is still possible to overwrite the defaults
    % using explicit options in \includegraphics[width, height, ...]{}
    \setkeys{Gin}{width=\maxwidth,height=\maxheight,keepaspectratio}
$endif$
$if(links-as-notes)$
    % Make links footnotes instead of hotlinks:
    \renewcommand{\href}[2]{#2\footnote{\url{#1}}}
$endif$
$if(strikeout)$
    \usepackage[normalem]{ulem}
    % avoid problems with \sout in headers with hyperref:
    \pdfstringdefDisableCommands{\renewcommand{\sout}{}}
$endif$

\usepackage[unicode=true]{hyperref}
\hypersetup{breaklinks=true,
            bookmarks=true,
            pdfauthor={$author-meta$},
            pdftitle={$title-meta$},
            colorlinks=true,
            citecolor=$if(citecolor)$$citecolor$$else$blue$endif$,
            urlcolor=$if(urlcolor)$$urlcolor$$else$blue$endif$,
            linkcolor=$if(linkcolor)$$linkcolor$$else$black$endif$,
            pdfborder={0 0 0}}
\urlstyle{same}

$if(verbatim-in-note)$
    \VerbatimFootnotes % allows verbatim text in footnotes
$endif$
\providecommand{\tightlist}{%
  \setlength{\itemsep}{0pt}\setlength{\parskip}{0pt}}

\usepackage{upquote}

\usepackage{titlesec}
\titleformat{\chapter}{\normalfont\huge\bfseries}{\thechapter.}{20pt}{\huge\bf}
% \titleclass{\section}{top}
% \newcommand\sectionbreak{\clearpage}

\pagestyle{plain}

%% This bit allows you to either specify only the files which you wish to
%% process, or `all' to process all files which you \include.
%% Krishna Sethuraman (1990).

% \typein [\files]{Enter file names to process, (chap1,chap2 ...), or `all' to
% process all files:}
% \def\all{all}
% \ifx\files\all \typeout{Including all files.} \else \typeout{Including only \files.} \includeonly{\files} \fi

\begin{document}


% NOTE:
% These templates make an effort to conform to the MIT Thesis specifications,
% however the specifications can change.  We recommend that you verify the
% layout of your title page with your thesis advisor and/or the MIT
% Libraries before printing your final copy.
\title{$title$}

\author{$author$}
\prevdegrees{S.B., Massachusetts Institute of Technology (2013)}
% If you wish to list your previous degrees on the cover page, use the
% previous degrees command:
%       \prevdegrees{A.A., Harvard University (1985)}
% You can use the \\ command to list multiple previous degrees
%       \prevdegrees{B.S., University of California (1978) \\
%                    S.M., Massachusetts Institute of Technology (1981)}
\department{Department of Electrical Engineering and Computer Science}

% If the thesis is for two degrees simultaneously, list them both
% separated by \and like this:
% \degree{Doctor of Philosophy \and Master of Science}
\degree{Master of Engineering in Computer Science and Engineering}

% As of the 2007-08 academic year, valid degree months are September,
% February, or June.  The default is June.
\degreemonth{February}
\degreeyear{2016}
\thesisdate{January 29, 2016}

%% By default, the thesis will be copyrighted to MIT.  If you need to copyright
%% the thesis to yourself, just specify the `vi' documentclass option.  If for
%% some reason you want to exactly specify the copyright notice text, you can
%% use the \copyrightnoticetext command.
%\copyrightnoticetext{\copyright IBM, 1990.  Do not open till Xmas.}

% If there is more than one supervisor, use the \supervisor command
% once for each.
\supervisor{Joshua B. Tenenbaum}{Professor}

% This is the department committee chairman, not the thesis committee
% chairman.  You should replace this with your Department's Committee
% Chairman.
\chairman{Christopher Terman}{Chairman, Masters of Engineering Thesis Committee}

% Make the titlepage based on the above information.  If you need
% something special and can't use the standard form, you can specify
% the exact text of the titlepage yourself.  Put it in a titlepage
% environment and leave blank lines where you want vertical space.
% The spaces will be adjusted to fill the entire page.  The dotted
% lines for the signatures are made with the \signature command.
\maketitle

% The abstractpage environment sets up everything on the page except
% the text itself.  The title and other header material are put at the
% top of the page, and the supervisors are listed at the bottom.  A
% new page is begun both before and after.  Of course, an abstract may
% be more than one page itself.  If you need more control over the
% format of the page, you can use the abstract environment, which puts
% the word "Abstract" at the beginning and single spaces its text.

%% You can either \input (*not* \include) your abstract file, or you can put
%% the text of the abstract directly between the \begin{abstractpage} and
%% \end{abstractpage} commands.

% First copy: start a new page, and save the page number.
\cleardoublepage
% Uncomment the next line if you do NOT want a page number on your
% abstract and acknowledgments pages.
% \pagestyle{empty}
\setcounter{savepage}{\thepage}
\begin{abstractpage}
$abstract$
\end{abstractpage}

% Additional copy: start a new page, and reset the page number.  This way,
% the second copy of the abstract is not counted as separate pages.
% Uncomment the next 6 lines if you need two copies of the abstract
% page.
% \setcounter{page}{\thesavepage}
% \begin{abstractpage}
% \input{abstract}
% \end{abstractpage}

\cleardoublepage



\section*{Acknowledgments}
\section{Acknowledgements}\label{acknowledgements}

To my own sweet ass, for all its hard work.

%
% This is the acknowledgements section.  You should replace this with your
% own acknowledgements.

%%%%%%%%%%%%%%%%%%%%%%%%%%%%%%%%%%%%%%%%%%%%%%%%%%%%%%%%%%%%%%%%%%%%%%
% -*-latex-*-

$for(include-before)$
    $include-before$
    \pagestyle{plain}
$endfor$

$if(toc)$
    {
    % \hypersetup{linkcolor=black}
    \setcounter{tocdepth}{$toc-depth$}
    \tableofcontents
    }
$endif$
$if(lot)$
    \listoftables
$endif$
$if(lof)$
    \listoffigures
$endif$
% Some departments (e.g. 5) require an additional signature page.  See
% signature.tex for more information and uncomment the following line if
% applicable.
% \include{signature}

$body$

\begin{singlespace}

\end{singlespace}
$if(natbib)$
    $if(bibliography)$
        $if(biblio-title)$
            $if(book-class)$
                \renewcommand\bibname{$biblio-title$}
            $else$
                \renewcommand\refname{$biblio-title$}
            $endif$
        $endif$
        \bibliography{$for(bibliography)$$bibliography$$sep$,$endfor$}

    $endif$
$endif$
$if(biblatex)$
    \printbibliography$if(biblio-title)$[title=$biblio-title$]$endif$

    \bibliographystyle{plain}
    \end{singlespace}
$endif$
$for(include-after)$
    $include-after$
$endfor$

% \include{contents}
% \include{chap1}
% \include{chap2}
% \appendix
% \include{appa}
% \include{appb}
% \include{biblio}
\end{document}
